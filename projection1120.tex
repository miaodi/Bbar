\documentclass{article}
\renewcommand{\labelenumii}{\theenumii}
\renewcommand{\theenumii}{\theenumi.\arabic{enumii}.}
\newenvironment{rcases}
  {\left.\begin{aligned}}
  {\end{aligned}\right\rbrace}

\usepackage{amssymb,amsmath,amsthm,mathrsfs}
\usepackage{multirow}
\usepackage{tabularx}
\usepackage{diagbox}
\usepackage{cite}
\usepackage{indentfirst}
\usepackage{graphicx,subcaption}
\usepackage[table]{xcolor}
\usepackage{standalone}
\usepackage{relsize}
\usepackage{structuralanalysis}
\usepackage{changes}
\graphicspath{ {images/} }
\usepackage{tikz,array,calc}
\usetikzlibrary{decorations.pathreplacing}
\definecolor{green1}{RGB}{217,240,211}
\newtheorem{remark}{Remark}
\newtheorem{lemma}{Lemma}
\usepackage{xfrac}


\DeclareMathOperator*{\A}{ \mathlarger{\mathlarger{\mathlarger{\boldsymbol{\mathsf{A}}}}} }
\DeclareMathOperator*{\assembly}{{\scalerel*{\text{\sf \Huge A}}{\sum}}}
\DeclareMathOperator*{\assemblyt}{{\scalerel*{\text{\sf \Huge A}}{\textstyle\sum}}}
\newcommand\vx{\mathbf{x}}
\newcommand\vy{\mathbf{y}}
\newcommand\vz{\mathbf{z}}
\newcommand\vu{\mathbf{u}}
\newcommand\vv{\mathbf{v}}
\newcommand\vg{\mathbf{g}}
\newcommand\ve{\mathbf{e}}
\newcommand\ma{\mathbf{A}}
\newcommand\mi{\mathbf{I}}
\newcommand\mn{\mathbf{N}}
\newcommand{\real}{\mathbb{R}}
\newcommand{\rn}{\mathbb{R}^d}
\newcommand{\mat}[1]{\boldsymbol{\mathbf{#1}}}
\newcommand{\parent}{\tilde{\Omega}}
\newcommand{\param}{\hat{\Omega}}
\newcommand{\diag}[1]{\textrm{diag}(#1)}
\newcommand{\dualop}[1]{\mat{D}_{#1}}
\newcommand{\paramdualop}{\hat{\mat{D}}}
\newcommand{\prntdualop}{\tilde{\mat{D}}}
\newcommand{\dualfunc}[1]{\bar{#1}}
\newcommand{\ratdualfunc}[2]{\bar{R}^{#1}_{#2}}
\newcommand{\prntdualfunc}[1]{{\tilde{\bar{#1}}}}
\newcommand{\paramdualfunc}[1]{\hat{\bar{#1}}}
\newcommand{\spatdualfunc}[1]{{\bar{\bar{#1}}}}
\newcommand{\nosum}{\,\textrm{(no sum)}}
\newcommand{\vol}[1]{\,\textrm{vol}(#1)}
\newcommand{\spatial}{\Omega}
\renewcommand{\vec}[1]{\boldsymbol{\mathbf{#1}}}
\newcommand{\stkout}[1]{\ifmmode\text{\sout{\ensuremath{#1}}}\else\sout{#1}\fi}
\setdeletedmarkup{\stkout{#1}}
\usepackage{geometry}
 \geometry{
 a4paper,
 total={170mm,257mm},
 left=20mm,
 top=20mm,
 }
 
 
\newcommand{\Bezier}{{B\'{e}zier} }
\renewenvironment{abstract}{%
\hfill\begin{minipage}{0.95\textwidth}
\rule{\textwidth}{1pt}}
{\par\noindent\rule{\textwidth}{1pt}\end{minipage}}
\providecommand{\keywords}[1]{\textbf{\textit{Keywords: }}#1}

\title{B\'{e}zier $\bar{B}$ projection}
\author{Di Miao$^{1,}$\thanks{miaodi1987@gmail.com} \and Michael J. Borden$^1$ \and Michael A. Scott$^1$ \and Derek C. Thomas$^2$}
\date{}

\begin{document}
\maketitle
\vspace{-2em}
% * <miaodi1987@gmail.com> 2017-09-06T02:28:28.868Z:
%
% ^.
\begin{center}
%% $^1$ Civil, Construction, and Environmental Engineering\\
%% North Carolina State University \\
%% Box 7908, Raleigh, North Carolina 27695 , USA\\ \vspace{1em}

$^1$Department of Civil and Environmental Engineering\\
Brigham Young University\\
368 CB, Provo, UT 84602, USA\\ \vspace{1em}
 
$^2$Coreform LLC\\
P.O. Box 970336, Orem, UT 84097, USA \\ 
\end{center}

%\textbf{Contributions}
%\begin{itemize}
%\item Develop a local $\bar{B}$ method based on B\'ezier projection.
%\item Present element level algorithms based on \Bezier extraction to show how the method can be implemented into a finite element code.
%\item Numerically demonstrate that transverse shear locking and volumetric locking can be alleviated in the context of Timoshenko beams and nearly incompressible elasticity, respectively.
%\end{itemize}

\begin{abstract}
  We demonstrate the use of B\'{e}zier projection to alleviate locking phenomena in structural mechanics applications of isogeometric analysis. \added{Based on two different 	
  interpretations of well-known $\bar{B}$ projection, we propose two formulations and name them (symmetric/non-symmetric) B\'{e}zier $\bar{B}$ projection.} To demonstrate the utility of the \replaced{\Bezier projection}{approach} for both geometry and material locking phenomena we focus on transverse shear locking in Timoshenko beams and volumetric locking in nearly compressible linear elasticity although the approach can be applied generally to other types of locking phenemona as well. B\'{e}zier projection is a local projection technique with optimal approximation properties, which in many cases produces solutions that are comparable to global $L^2$ projection. In the context of $\bar{B}$ methods, the use of B\'ezier projection produces sparse stiffness matrices with only a slight increase in bandwidth when compared to standard displacement-based methods. Of particular importance is that the approach is applicable to any spline representation that can be written in B\'ezier form like NURBS, T-splines, LR-splines, etc. We discuss in detail how to integrate this approach into an existing finite element framework with minimal disruption through the use of B\'ezier extraction operators and a newly introduced dual \added{basis for} \Bezier \replaced{projection}{extraction} operator. We then demonstrate the behavior of the \replaced{two proposed formulations}{approach} through several challenging benchmark problems.
\end{abstract}
\keywords{Isogeometric analysis, B\'ezier extraction, \added{\Bezier dual basis}, B\'ezier projection, $\bar{B}$-projection, locking}
\section{Introduction}
Isogeometric analysis (IGA), introduced by Hughes et al. \cite{hughes_isogeometric_2005}, adopts the spline basis, which underlies the CAD geometry, as the basis for analysis. Of particular importance is the positive impact of smoothness on numerical solutions, where, in many application domains, IGA outperforms classical finite elements~\cite{cottrell_isogeometric_2009,cottrell_studies_2007,cottrell2006isogeometric,hughes_duality_2008,bazilevs_isogeometric_2010,evans_n-widths_2009}. Initial investigations of IGA focused on non-uniform rational B-splines (NURBS) due to their dominance in commercial CAD packages. However, many advances are being made in analysis-suitable geometry representations that overcome the strict rectangular topological restrictions of NURBS. Examples include T-splines~\cite{bazilevs_isogeometric_2010,sederberg_t-splines_2003} and their analysis-suitable restriction~\cite{scott_local_2012, li_analysis-suitable_2013}, hierarchical B-splines~\cite{bornemann_subdivision-based_2013,scott_isogeometric_2014,schillinger_isogeometric_2012,evans_hierarchical_2015,forsey_hierarchical_1988}, and locally refined B-splines~\cite{dokken_polynomial_2013,johannessen_isogeometric_2014} among others.

The purpose of this paper is to demonstrate how B\'{e}zier projection~\cite{thomas_bezier_2015} can be employed as the underlying local projection framework for a $\bar{B}$ approach to treat locking in isogeometric structural elements. B\'ezier projection is an element-based local projection methodology for B-splines, NURBS, and T-splines. It relies on the concept of B\'ezier extraction~\cite{borden_isogeometric_2011, scott_isogeometric_2011} and an associated operation, spline reconstruction, which enables the use of B\'ezier projection in standard finite element codes.

B\'ezier projection exhibits provably optimal convergence and yields
projections that are virtually indistinguishable from global $L^2$ projection. For an isogeometric finite element code that leverages B\'ezier extraction, B\'ezier projection can be employed virtually for free. To simplify the implementation of the \Bezier $\bar{B}$ method in existing finite element codes we develop a \textit{dual} element \Bezier extraction operator that can be derived directly from the \Bezier extraction of a spline representation. It is worth noting that B\'ezier projection can also be used to develop a unified framework for spline operations including cell subdivision and merging, degree elevation and reduction, basis roughening and smoothing, and spline reparameterization and is applicable to any spline representation that can be written in B\'ezier form.

Numerical locking in structural finite elements includes geometric locking in thin curved structural members such as membrane and shear locking and also includes volumetric locking in incompressible and nearly incompressible elasticity. There is an immense body of literature on approaches to overcome locking in the finite element community and various approaches have emerged as dominant. These include reduced quadrature~\cite{malkus_mixed_1978,zienkiewicz_reduced_1971}, $\bar{B}$ projection methods~\cite{nagtegaal_numerically_1974,hughes_generalization_1980}, and mixed methods based on the Hu-Washizu variational principle~\cite{dolbow_volumetric_1999,kasper_mixed-enhanced_2000,hughes_variational_1986}. It is important to mention that, although ameliorated at high polynomial degrees, smooth splines in the context of IGA still exhibit locking behavior~\cite{echter_numerical_2010, bouclier_locking_2012}.

In IGA, there is a growing literature on the treatment of locking in structural elements. Leveraging higher-order smoothness, transverse shear locking can be eliminated at the theoretical level by employing Kirchhoff-Love~\cite{kiendl_isogeometric_2009, kiendl_isogeometric_2015} and hierarchic Reissner-Mindlin~\cite{oesterle_hierarchic_2017, oesterle_shear_2016,echter_hierarchic_2013} shell elements. Reduced quadrature schemes have been explored in~\cite{adam_improved_2014, adam_improved_2015, adam_selective_2015} as a way to alleviate transverse shear locking. The extension of $\bar{B}$ projection to the isogeometric setting was initiated in~\cite{elguedj:hal-00457010} for both elastic and plastic problems and was extended in~\cite{bouclier_efficient_2013} to include local projection techniques~\cite{mitchell_method_2011,govindjee_convergence_2012}.

The outline of this paper is as follows: First, we briefly review spline basis functions in Section~\ref{sec:preliminaries}. In Sections~\ref{sec:extraction} and~\ref{sec:bproject}, we describe B\'ezier extraction and projection. We then formulate and use B\'ezier $\bar{B}$ projection for the Timoshenko beam (to treat transverse shear locking) and nearly incompressible elasticity (to treat volumetric locking) in Sections~\ref{sec:timoshenkobeam} and~\ref{sec:nearlyincompressible}, respectively. We provide detailed element level operations in both settings. We also presents numerical tests to show the performance of the proposed strategy. %We then conclude in Section~\ref{sec:conclusion}.

\section{Preliminaries and notation}
\label{sec:preliminaries}
In this section a brief overview of univariate Bernstein, B-spline, and NURBS basis functions is provided. We also describe how these univariate basis functions are extended to higher dimensions.
\subsection{Univariate Bernstein basis functions}
The $i$th univariate Bernstein basis function of degree $p$ is defined by
\begin{equation}
B_{i,p}(\xi)=\binom {p}{i}\xi^i(1-\xi)^{n-i}
\end{equation}
where $\xi\in\left[ 0,1 \right]$ and $\binom {p}{i}=\dfrac{p!}{i!(n-i)!}$, $0\leq{i}\leq{p}$, is a binomial coefficient.
\subsection{Univariate spline basis functions}
A univariate B-spline basis of dimension $n$ is defined by a polynomial degree $p$ and a knot vector $\mathbf{\Xi}=\lbrace{\xi_0,\xi_1,\ldots, \xi_{n+p}}\rbrace$, which is a non-decreasing sequence of real numbers. The $A$th B-spline basis function can then be defined using the Cox-de Boor recursion formula:
\begin{gather}
N_{A,0}(\xi)=\begin{cases}1 & \xi_A\leq{\xi}\leq{\xi_{A+1}}\\0 & otherwise \end{cases} \\
N_{A,p}(\xi)=\dfrac{\xi-\xi_A}{\xi_{A+p}-\xi_A}N_{A,p-1}(\xi)+\dfrac{\xi_{A+p+1}-\xi}{\xi_{A+p+1}-\xi_{A+1}}N_{A+1,p-1}(\xi).
\end{gather}\par
For simplicity, we will always use open knot vectors defined over the interval $\left[ 0,1\right]$. An open knot vector satisfies the conditions $\xi_0=\xi_1=\dots=\xi_{p}=0$ and $\xi_{n}=\xi_{n+1}=\dots=\xi_{n+p}=1$ and creates interpolatory end conditions. B-spline basis functions can be used to represent piecewise polynomial functions but are not capable of representing conic sections (e.g. circles, ellipses and hyperbolas). NURBS overcome this shortcoming. A NURBS basis function can be written as
\begin{equation}
R_{A,p}(\xi)=\dfrac{N_{A,p}(\xi)w_A}{W{(\xi)}}
\end{equation}
where $w_A$ is called a weight and
\begin{align}
  \label{eq:weight}
W(\xi)=\sum_{A} N_{A,p}(\xi)w_A
\end{align}
is called the weight function. A $d$-dimensional rational curve $\mathbf{S}(\xi)\in{\rn}$ can then be defined as
\begin{equation}
\mathbf{S}(\xi)=\sum_A R_{A,p}(\xi)\mathbf{P}_A
\end{equation}
where $\mathbf{P}_A=(p_A^1,p_A^2,\ldots,p_A^d)^T$ is a $d$-dimensional control point. It is often more convenient to represent the $d$-dimensional NURBS in a $(d+1)$-dimensional homogeneous space by defining $\mathbf{P}_A^w=(p_A^1w_A,p_A^2w_A,\ldots,p_A^dw_A,w_A)^T$ and the corresponding $(d+1)$-dimensional B-spline curve as
\begin{align}
\mathbf{S}^w(\xi)&=\sum_A N_{A,p}(\xi)\mathbf{P}_A^w
\end{align}
such that each component of $\mathbf{S}^w$ can be written as
\begin{align}
S_i(\xi)&=\dfrac{{S}_i^w(\xi)}{{S}_{d+1}^w(\xi)}.
\end{align}
In the homogeneous form, NURBS can be manipulated with standard B-spline algorithms.

\subsection{Multivariate spline basis functions}
In higher dimensions, Bernstein, B-spline, and NURBS basis functions are formed by the Kronecker product of univariate basis functions. For example, two-dimensional B-spline basis functions of degree $\mathbf{p}=(p_\xi, p_\eta)$ are defined by
\begin{equation}
\mathbf{N}^\mathbf{p}(\xi,\eta)=\mathbf{N}^{p_\xi}(\xi)\otimes\mathbf{N}^{p_\eta}(\eta)
\end{equation}
where $\mathbf{N}^{p_\xi}(\xi)$ and $\mathbf{N}^{p_\eta}(\eta)$ are vectors of basis functions in the $\xi$ and $\eta$ directions, respectively. A particular multivariate basis function can be written as
\begin{equation}
{N}_{A(i,j)}^\mathbf{p}(\xi,\eta)={N}_{i,p_\xi}(\xi){N}_{j,p_\eta}(\eta)
\end{equation}
where the index mapping is defined as
\begin{equation}
A(i,j)=n_\eta{i}+j.
\end{equation}
The integer $n_\eta$ is the number of basis functions in $\eta$ direction.
\section{B\'{e}zier extraction}
\label{sec:extraction}
Given a spline basis $\mathbf{N}$ there exists a Bernstein basis $\mathbf{B}$ and a linear operator $\mathbf{C}$ (see~\cite{borden_isogeometric_2011}) such that
\begin{equation}
\mathbf{N}(\xi)=\mathbf{C}\mathbf{B}(\xi).
\end{equation}
The localization of $\mathbf{C}$ to an element domain produces the element extraction operator $\mathbf{C}^e$.
Given control points $\mathbf{P}^e$, the corresponding B\'ezier control points $\mathbf{Q}^e$ can be computed directly as
\begin{equation}
\mathbf{Q}^e=(\mathbf{C}^e)^T\mathbf{P}^e.
\end{equation}
A graphical depiction of B\'{e}zier extraction is shown in Figure~\ref{fig:extraction_and_projection}.

\begin{figure}
  \centering
  \includegraphics[width=.6\linewidth]{elements_global.pdf}
  \caption{Illustration of B\'ezier extraction and projection in one dimension for a B-spline of degree 2 and knot vector $[0,0,0,1/3/2/3,1,1,1]$ (restricted to the second element for illustrative purposes).}
    \label{fig:extraction_and_projection}
\end{figure}
\section{B\'ezier projection}
\label{sec:bproject}

B\'{e}zier projection can be viewed as the inverse of extraction~\cite{thomas_bezier_2015}. B\'ezier projection uses an element reconstruction operator $\mathbf{R}^e\equiv(\mathbf{C}^e)^{-1}$ such that the global control point values, corresponding to those basis functions defined over the support of an element $e$, can be determined directly from \Bezier control values as
\begin{equation}
\mathbf{P}^e=(\mathbf{R}^e)^T\mathbf{Q}^e
\end{equation}
where $\mathbf{Q}^e$ is any field in B\'ezier form. The action of the element reconstruction operator is depicted graphically in Figure~\ref{fig:extraction_and_projection}. For example, given any function $u \in L^2$, we can compute $\mathbf{Q}^e$ as
\begin{equation}
\mathbf{Q}^e=(\mathbf{G}^e)^{-1}\mathbf{F}^e
\label{eq:element-Qi}
\end{equation}
where $\mathbf{G}^e$ is the Gramian matrix corresponding to the Bernstein basis with components
\begin{align}
  {G}_{ij}^e &= \int_{\Omega^e} B^e_i B^e_j \, d\Omega =\langle{B^e_{i},B^e_{j}}\rangle_{\Omega^e}
\end{align}
and
\begin{align}
  {F}^e_i &=  \int_{\Omega^e} B^e_i u \, d\Omega = \langle{B^e_{i,},u}\rangle_{\Omega^e}.
\end{align}
Note that efficiency gains can be had at the expense of accuracy by instead performing the integration in the parametric domain of the element~\cite{thomas_bezier_2015}. 

The element-wise projection produces one control value for each element in the support of the function.  These values must be combined in order to provide the final control value.  A core component of the B\'ezier projection algorithm is the definition of an appropriate averaging operation.  A weighted average of the values is computed using the weighting
\begin{equation}\label{eqn:Bezier_weight}
\omega_a^e=\dfrac{\int_{\Omega^e} N_{a}^e \, d\Omega}{\int_{\Omega^A} N_{A(e,a)} \, d\Omega}
\end{equation}
where $\Omega^e$ corresponds to the physical domain of element $e$, $A(e,a)$ is a mapping from a local nodal index $a$ defined over element $e$ to a corresponding global node index $A$, and $\Omega^A$ corresponds to the physical support of $N_A$. The final averaged global control point is then calculated as
\begin{equation}
P_A=\sum_{\Omega^e\in \Omega^A } \omega_{A(e,a)} P_{A(e,a)}.
\end{equation}
B\'ezier projection onto NURBS functions can be defined in an analogous manner~\cite{thomas_bezier_2015}.

The individual steps comprising the \Bezier projection algorithm are
illustrated in Figure~\ref{fig:loc-proj-example} where
the curve defined by $\vec{f}(t)=\left( \frac{t}{3}
\right)^{3/2}\vec{e}_1+\frac{1}{10}\sin (\pi t )\,\vec{e}_2$,
$t\in[0,3]$ is projected onto the quadratic B-spline basis defined by
the knot vector $[0,0,0,\sfrac{1}{3},\sfrac{2}{3},1,1,1]$. For this example,
the algorithm proceeds as follows:

\begin{description}

\item{Step 1:} The function $\mathbf{f}$ is projected onto the Bernstein basis of each element. This results in a set of
  \Bezier coefficients that define an approximation to $\mathbf{f}$.
  The \Bezier coefficients are indicated in part (1) of Figure~\ref{fig:loc-proj-example} by 
  square markers that have been colored to match the corresponding
  element. Each \Bezier segment is discontinuous.

\item{Step 2:} The element reconstruction operator $\mathbf{R}^e$ is used to convert the
  \Bezier control points into spline control points associated with the
  basis function segments over each element.
  The new control points are marked with inverted triangles and
  again colored to indicate the element with which the control point is
  associated. The control points occur in clusters.
  The clusters of control points represent the contributions from
  multiple elements to a single spline basis function control point.

\item{Step 3:} Each cluster of control points is averaged to obtain a
  single control point by weighting each point in the cluster according
  to the weighting given in (\ref{eqn:Bezier_weight}). The resulting control
  points are shown as circles with the relative contribution from each
  element to each control point indicated by the colored fraction of the
  control point marker. e colors in Figures~\ref{fig:weights} and~\ref{fig:loc-proj-example} are coordinated
  to illustrate where the averaging weights come from and their values.
\end{description}

\begin{figure}[htb]
  \centering
  \includegraphics[width=5in]{weights.pdf}
  \caption{\label{fig:weights}Weights over each knot span associated
    with the basis function defined by the knot vector
    $[0,0,0,\sfrac{1}{3},\sfrac{2}{3},1,1,1]$.} 
\end{figure}
\begin{figure}[htb]
  \centering
  \begin{tabular}{c p{4in} p{2in}}
    (0)&\raisebox{-.5\height}{\includegraphics[width=4in]{local_proj_steps1.pdf}} & \begin{minipage}[t]{2in}Target function\end{minipage}\\
    (1)&\raisebox{-.5\height}{\includegraphics[width=4in]{local_proj_steps2.pdf}} & \begin{minipage}[t]{2in}Perform local projection to obtain \Bezier control points (represented by squares, colored to match elements)\end{minipage}\\
    (2)&\raisebox{-.5\height}{\includegraphics[width=4in]{local_proj_steps3.pdf}} & \begin{minipage}[t]{2in}Use element reconstruction operator to project \Bezier points to spline control points (represented by inverted triangles, colored to match elements)\end{minipage}\\
    (3)&\raisebox{-.5\height}{\includegraphics[width=4in]{local_proj_steps4.pdf}} & \begin{minipage}[t]{2in}Apply smoothing algorithm (contribution of each element to each control point shown by colored fraction)\end{minipage}\\
    (4)&\raisebox{-.5\height}{\includegraphics[width=4in]{local_proj_steps5.pdf}} & \begin{minipage}[t]{2in}Comparison of final function (light blue) and target function (dashed)\end{minipage}
  \end{tabular}
  \caption{\label{fig:loc-proj-example}Steps of \Bezier projection.}
\end{figure}

\subsection{Dual basis formulation of B\'{e}zier projection}
\label{sec:dual-funct-dual}
To integrate \Bezier projection into a standard finite element assembly algorithm, it is convenient to recast \Bezier projection in terms of a dual basis. \added{This approach plays a key role in developing the non-symmetric \Bezier $\bar{B}$ method discussed in Section~\ref{sec:non-symmetric}.} A dual basis has the distinguishing property that
\begin{align}
  \int_{\Omega} \hat{N}_A N_B \, d\Omega = \delta_{AB}.
  \end{align}
Once a dual basis is defined it can be processed in much the same manner as standard basis functions are processed in a finite element code. A complete exposition on the subject of dual bases and the \Bezier projection framework can be found in~\cite{thomas_bezier_2015}. We first define the dual element extraction operator
\begin{align}
 \hat{\mathbf{D}}^e=\operatorname{diag}({\boldsymbol{\omega}^e})\mathbf{R}^e(\mathbf{G}^e)^{-1}
\end{align}
where $\mathbf{G}^e$ is the Gramian matrix of the Bernstein basis functions over the element and $\operatorname{diag}({\boldsymbol{\omega}^e})$ is a diagonal matrix that contains the B\'ezier projection weights computed by (\ref{eqn:Bezier_weight}). We can then define a dual basis function $\hat{N}_{A(e,a)}$ restricted to element $e$ as
\begin{align}
\hat{N}^e_{a} = \sum_{j}\hat{D}_{aj}^e B_{j}.
\label{eq:local-dual-basis}
\end{align}

The biorthogonality of the dual basis can be seen by noting that
\begin{align}
  \int_{\Omega^e } \hat{\mathbf{N}}^{e} (\mathbf{N}^{e})^T \, d\Omega &= \operatorname{diag}(\boldsymbol{\omega}^e)
\end{align}
and
\begin{align}
\A_{e} \left[\int_{\Omega^e } \hat{\mathbf{N}}^{e} (\mathbf{N}^{e})^T \, d\Omega \right] &= \mathbf{I}
\end{align}
where $\A$ is the standard finite element assembly operator \cite{hughes_finite_2012}.

Now, given any function $u \in L^2$ we can use the dual basis to find its representation in terms of the corresponding spline basis as
\begin{align}
  u &= \sum_A P_A N_A
\end{align}
where
\begin{align}
	P_A = \int_{\Omega^A} \hat{N}_{A} u \, d\Omega = \langle \hat{N}_A, u \rangle_{\Omega^A}.
\end{align}
A set of dual basis functions corresponding to the quadratic maximally smooth B-spline basis shown in Figure~\ref{fig:dualbasis}a is shown in Figure~\ref{fig:dualbasis}c. Note that these dual functions have compact support and discontinuities which coincide with the underlying knots in the knot vector. The compact support of the dual basis functions will be crucial for maintaining the sparsity of the stiffness matrix for the B\'{e}zier $\bar{B}$ formulations presented in this paper. For comparison, the dual basis corresponding to \textit{global} $L^2$ projection are shown in Figure~\ref{fig:dualbasis}b. Each of these dual basis functions has global support which explains why the use of global $\bar{B}$ projections results in dense stiffness matrices. 
\begin{figure}
    \centering
    \begin{subfigure}[b]{0.32\linewidth}        %% or \columnwidth
        \centering
        \includegraphics[width=\linewidth]{basisfuntioncs}
        \caption{}
    \end{subfigure}
    \begin{subfigure}[b]{0.32\linewidth}        %% or \columnwidth
        \centering
        \includegraphics[width=\linewidth]{dualGlobal}
        \caption{}
    \end{subfigure}
    \begin{subfigure}[b]{0.32\linewidth}        %% or \columnwidth
        \centering
        \includegraphics[width=\linewidth]{dualbasis}
        \caption{}
    \end{subfigure}
    \caption{A set of quadratic B-spline basis functions (a), and corresponding dual basis functions computed by a global $L^2$ projection (b) and the dual basis functions computed using B\'ezier projection (c). Note that the support of the dual basis functions in (b) is not compact. The dual basis functions shown in (c) have compact support but are not continuous.}
    \label{fig:dualbasis}
\end{figure}

\subsubsection{Rational dual basis functions}
If rational basis functions are used, the construction of the dual basis must be modified slightly. A rational dual basis must satisfy the biorthogonality requirement
\begin{align}
  \int_{\Omega} \bar{R}_A R_B \, d\Omega= \delta_{AB}.
\end{align}
A simple way to achieve biorthogonality is to define
\begin{align}
  \bar{R}_A &= W \bar{N}_A
\end{align}
where $W$ is the rational weight given in (\ref{eq:weight}). Now
\begin{align}
  \int_{\Omega} \bar{R}_A R_B \, d\Omega &= \int_{\Omega} \bar{N}_A N_B \, d\Omega = \delta_{AB}.
  \end{align}

\section{Geometric locking: Timoshenko beams}
\label{sec:timoshenkobeam}
To illustrate the use of \Bezier $\bar{B}$ projection to overcome geometric locking effects we study transverse shear locking in Timoshenko beams. The Timoshenko beam problem provides a simple one dimensional setting in which to describe B\'ezier $\bar{B}$ projection. Note however, that the approach can be directly generalized to more complex settings like spatial beams and shells and other geometric locking mechanisms like membrane locking. We consider a planar cantilevered Timoshenko beam as shown in Figure~\ref{fig:Timoshenko_beam_cross}. The strong form for this problem can be stated as
\begin{align}
\begin{rcases}
\begin{aligned}
  -s{GA}\gamma' &=f(x) \\
  -EI\kappa'-s{GA}\gamma &= 0\\
  \kappa &= \phi' \\
  \gamma &= \omega'-\phi
\end{aligned}
\end{rcases}
& \text{ in $\Omega$}\\
\begin{rcases}
\begin{aligned}
\omega&=0\\
\phi&=0
\end{aligned}
\end{rcases}
&\text{ at $x=0$}\\
\begin{rcases}
\begin{aligned}
s{AG}\gamma &= Q\\
-EI\kappa &= M
\end{aligned}
\end{rcases}
&\text{ at $x=L$}
\end{align}
where $\gamma$ is the shear strain, $\kappa$ is the bending strain, $\omega$ is the vertical displacement, $\phi$ is the angle of rotation of the normal to the mid-plane of the beam, $f$ is the distributed transverse load, $Q$ is a point load, $M$ is the moment, $E$ is the Young's modulus, $G$ is the shear modulus, $A$ is the cross-sectional area, $I$ is the second moment of inertia of the beam cross-section, $s$ is the shear correction factor, normally set to $5/6$ for rectangular cross-sections, and $\Omega = (0,L)$. When $\omega$ and $\phi$ are interpolated by basis functions of the same order the finite element solution to this problem exhibits shear locking as the beam becomes slender.

\begin{figure}[ht]
  \centering
  \includegraphics[width=0.25\linewidth]{timoshenko_beam_shape}
  \caption{Deformation of a Timoshenko beam. The normal rotates by the angle $\phi$, which is not equal to $w'$, due to shear deformation.}
  \label{fig:Timoshenko_beam_cross}
\end{figure}

\subsection{Symmetric \Bezier $\bar{B}$ projection}
\label{sec:symmetric-projection}
{\color{blue}For the Timoshenko beam, it is the transverse shear that causes locking. Hence, under the framework of $\bar{B}$ projection, the shear strain $\gamma$ is projected onto a lower order function space. The projected shear strain $\bar{\gamma}$ is used when solving the strain energy minimization problem. Here we use the \Bezier projection operator to obtain $\bar{\gamma}$, and refer to it as the Symmetric \Bezier $\bar{B}$ projection method because the resulting stiffness matrix is symmetric.}

\subsubsection{The weak form}

\sloppy Given the function spaces $\mathcal{S}(\Omega)=\{{\mathbf{u} \, \vert \, {\mathbf{u}\in{H^1(\Omega)\times{H^1(\Omega)}}},\mathbf{u}\vert_{\Gamma_{g}}=\mathbf{g}}\}$ and $\mathcal{V}(\Omega)=\{{\mathbf{w} \, \vert \, {\mathbf{w}\in{H^1(\Omega)\times{H^1(\Omega)}}}, \allowbreak \mathbf{w}\vert_{\Gamma_{g}}=\mathbf{0}}\}$ where $\mathbf{u}=\left\{{\omega,\phi}\right\}^T$, $\mathbf{w}=\left\{{\delta\omega,\delta\phi}\right\}^T$, $\mathbf{g}$ is the prescribed Dirichlet boundary condition, and $\Gamma_g$ is the Dirichlet boundary at $x=0$, the weak form of the problem can be stated as: find $\mathbf{u}\in{\mathcal{S}(\Omega)}$ such that for all $\mathbf{w}\in{\mathcal{V}(\Omega)}$
\begin{equation}
    {\int_{0}^L\kappa(\mathbf{w})EI\kappa(\mathbf{u}) + \bar{\gamma}(\mathbf{w})sGA\bar{\gamma}(\mathbf{u})} \, dx=\int_0^L\delta\omega f dx+\delta\omega(L)Q+\delta\phi(L)M.
\end{equation}
\deleted{and $\bar{\gamma}$ is the projected shear strain.}


\subsubsection{Discretization}
\replaced{We discretize $\omega$ and $\phi$ as}{We discretize $\mathbf{u}$ and $\mathbf{w}$ as}
\begin{align}
  \replaced{\omega = \sum_A \omega_A N_A}{\mathbf{u} = \sum_A \mathbf{U}_A N_A} \\
  \replaced{\phi = \sum_A \phi N_A}{\mathbf{u} = \sum_A \mathbf{U}_A N_A}
\end{align}
where \replaced{$N_A$ is a degree $p$ spline basis function, $\omega_A$ and $\phi_A$ are the corresponding control point values.}{$\mathbf{U}_A = \{\omega_A, \phi_A\}^T$ and $\mathbf{W}_A = \{\delta\omega_A, \delta\phi_A\}^T$ and $N_A$ is a degree $p$ spline basis function}. \added{The shear strain and bending strain can then be expressed as}
{\color{blue}\begin{align}
\gamma &= \sum_A{\begin{bmatrix}N'_A & -N_A\end{bmatrix}\begin{bmatrix}\omega_A & \phi_A\end{bmatrix}^T} \\
\kappa &= \sum_A\begin{bmatrix}0 & N'_A\end{bmatrix}\begin{bmatrix}\omega_A & \phi_A\end{bmatrix}^T
\end{align}}
The shear strain $\bar{\gamma}$ is constructed by B\'{e}zier projection of the true shear strain $\gamma$ onto a lower degree space. In other words, we project from a $p^{th}$ degree spline space with $n$ basis functions $\mathbf{N}$ defined by the knot vector
\begin{align}
\mathbf{\Xi}_{p}=\lbrace{\underbrace{0,0,\ldots,0}_\text{$p+1$ copies}},\mathbf{\Xi}_{int},{\underbrace{1,1,\ldots,1}_\text{$p+1$ copies}}\rbrace,
\label{eq:origin_knot_vector}
\end{align}
onto a $p-1^{th}$ degree spline space with $\bar{n}$ basis functions $\bar{\mathbf{N}}$ defined by the knot vector
\begin{equation}
\bar{\mathbf{\Xi}}_{p-1}=\lbrace{\underbrace{0,0,\ldots,0}_\text{$p$ copies}},\mathbf{\Xi}_{int},{\underbrace{1,1,\ldots,1}_\text{$p$ copies}}\rbrace
\label{eq:projected_knot_vector}
\end{equation}
where the internal knots, denoted by $\mathbf{\Xi}_{int}$, are the same for both spaces. The projected shear strain $\bar{\gamma}$ can then be written as
\begin{align}
  \bar{\gamma} &= \sum_A \bar{\gamma}_A \bar{N}_A.
\end{align}
The control variables $\bar{\gamma}_A$ are simply
\begin{align}
  \bar{\gamma}_A = \int_{\Omega^A} \hat{\bar{N}}_A \gamma \, d\Omega = \langle \hat{\bar{N}}_A, \gamma \rangle_{\Omega^i}
\end{align}
where $\hat{\bar{N}}_A$ is a dual basis function for the spline space of degree $p-1$ computed from \eqref{eq:local-dual-basis}.

Localizing to the B\'{e}zier element we define the strain-displacement arrays in terms of element Bernstein basis functions of degree $p$ and $p-1$ as
\begin{align}
	\mathbf{B}_e^\kappa 
    	&= \begin{bmatrix} 0 & -{B^e_{0,p}}' & \cdots & 0 & -{B^e_{p,p}}' \end{bmatrix}, \\
    \mathbf{B}_e^\gamma
    	&= \begin{bmatrix} {B^e_{0,p}}' & -B^e_{0,p} & \cdots & {B^e_{p,p}}' & -B^e_{p,p} \end{bmatrix}, \\
    \bar{\mathbf{B}}_e
    	&= \begin{bmatrix} \bar{B}^e_{0,p-1} & \cdots & \bar{B}^e_{p-1,p-1} \end{bmatrix},
\end{align}
\added{where $B^e_{i,p}$ is the $i^{th}$ Bernstein basis function of order $p$}. We can then compute the element arrays as
\begin{align}
\mathbf{K}_e^b &= EI\mathbf{C}^e\langle{{\mathbf{B}_e^\kappa}^T,\mathbf{B}_e^\kappa}\rangle(\mathbf{C}^e)^T, \label{eq:symmetric_timoshenko}\\
\bar{\mathbf{M}}_e &= s{GA}\bar{\mathbf{C}}^e\langle{\bar{\mathbf{B}}_e^T,\bar{\mathbf{B}}_e}\rangle(\bar{\mathbf{C}}^e)^T, \\
\replaced{\hat{\mathbf{P}}_e}{\mathbf{P}_e} &= \langle{(\hat{\bar{\mathbf{N}}}^e)^T,\mathbf{B}^\gamma_e}\rangle(\mathbf{C}^e)^T,
\end{align}
where $\mathbf{C}^e$ is the element extraction operator for the degree $p$ spline space, $\bar{\mathbf{C}}^e$ is the element extraction operator for the degree $p-1$ spline space, and $\hat{\bar{\mathbf{N}}}^e$ are the dual basis functions restricted to the element for the degree $p-1$ spline space. The global stiffness matrix can then be written as
\begin{align}
	\mathbf{K} = \mathbf{K}^b + \bar{\mathbf{K}}^s_S
\end{align}
where
\begin{align}
	\mathbf{K}^b &=\A_e\mathbf{K}_e^b,\label{eqn:bending_stiffness} \\
	\bar{\mathbf{K}}^s&=\replaced{\hat{\mathbf{P}}}{\mathbf{P}}^T\bar{\mathbf{M}}\replaced{\hat{\mathbf{P}}}{\mathbf{P}}\label{eqn:shear_stiffness} \\
	\replaced{\hat{\mathbf{P}}}{\mathbf{P}} &=\A_e\replaced{\hat{\mathbf{P}}_e}{\mathbf{P}_e} \\
    \bar{\mathbf{M}} &=\A_e\bar{\mathbf{M}}_e
\end{align}
and $\A$ is the standard finite element assembly operator \cite{hughes_finite_2012}. We note that the assembly of $\bar{\mathbf{K}}^s$ requires the assembly of two intermediate matrices, $\bar{\mathbf{M}}$ and $\replaced{\hat{\mathbf{P}}}{\mathbf{P}}$. The computation of these matrices is needed because the product of two integrals over the entire domain can not be localized to the element level.
\subsection{Non-symmetric \Bezier $\bar{B}$ projection}
\label{sec:non-symmetric}

\added{The relationship between $\bar{B}$-formulations and mixed formulations has been studied and it has been shown by Simo and Hughes \cite{hughes_variational_1986} that they are equivalent. However, the symmetric \Bezier $\bar{B}$ projection method presented in Section~\ref{sec:symmetric-projection}, where we started by considered the $\bar{B}$ formulation as a strain projection method, lacks this connection to mixed formulations (we will show this in the following section). In this section we present a second method based on \Bezier projection in which we view the $\bar{B}$ formulation as a mixed formulation with the auxiliary variable (the shear stress for the case of Timoshenko beam) being eliminated. Taking advantage of the dual basis formulation of the \Bezier projection operator we can incorporate \Bezier projection into the mixed formulation by using the \Bezier dual basis functions as the test functions of the auxiliary variable. Since the resulting stiffness matrix is not symmetric, this formulation is referred to as the non-symmetric \Bezier $\bar{B}$ projection method. Although this method does not preserve symmetry we will show that it does preserve convergence rates.}
\subsubsection{The weak form}
\added{In the mixed formulation, the shear stress $\tau=sGA\gamma$ is taken as a new independent variable. The weak form of the mixed formulation is then of the following form: find $\mathbf{u}\in{\mathcal{S}(\Omega)}$ and $\tau\in{L}^2(\Omega)$ such that for all $\mathbf{w}\in{\mathcal{V}(\Omega)}$ and $\delta\tau\in{L}^2(\Omega)$}
\begin{align}
    \int_{0}^L\kappa(\mathbf{w})EI\kappa(\mathbf{u}) + {\gamma}(\mathbf{w})\tau \, dx&= l\langle{\mathbf{w}}\rangle\\
    \int_{0}^L\delta\tau(sGA\gamma(\mathbf{u})-\tau) \, dx &= 0.
\end{align}
\subsubsection{Discretization}
\added{In the finite element formulation of the mixed form, the discretization of $\mathbf{u}$ and $\mathbf{w}$ remain the same. Since the shear strain and its variation are in $L^2(\Omega)$, their finite element approximation can be made from functions with poor regularity e.g. discontinuous polynomials. For consistency, if the field $\mathbf{u}$ is discretized by $p^{th}$ degree spline  basis functions defined by the knot vector \eqref{eq:origin_knot_vector}, the shear stress $\tau$ is then discretized by $p-1^{th}$ degree spline basis functions defined by the knot vector \eqref{eq:projected_knot_vector} and its variation $\delta\tau$ is discretized by the corresponding dual basis functions, as}
\begin{align}
    \tau&=\sum_A\tau_A\bar{N}_A \\
    \delta\tau&=\sum_A\delta\tau_A\hat{\bar{N}}_A
\end{align}
\added{The stiffness matrix for the mixed form can be written as}
\begin{equation}
   \mathbf{K}_{mix} =  
    \begin{bmatrix}
        \mathbf{K}^b & \mathbf{P}^T \\
        sGA\hat{\mathbf{P}} & -\mathbf{I}
    \end{bmatrix}
    \label{eq:mixed_timoshenko}
\end{equation}
\added{where}
\begin{equation}
    \mathbf{P}=\A_e\mathbf{P}_e,
\end{equation}
with
\begin{equation}
    \mathbf{P}_e=\bar{\mathbf{C}}^e\langle{\bar{\mathbf{B}}_e^T,\mathbf{B}^\gamma_e}\rangle(\mathbf{C}^e)^T.
\end{equation}
\added{The control variable of shear stress in \eqref{eq:mixed_timoshenko} can be eliminated, which leads to a pure displacement formulation as}
\begin{equation}
    \mathbf{K}=\mathbf{K}^b+\bar{\mathbf{K}}^s_{NS}, 
\end{equation}
where 
\begin{equation}
    \bar{\mathbf{K}}^s_{NS}=sGA\mathbf{P}^T\hat{\mathbf{P}}.
    \label{eq:nonsymmetric_timoshenko}
\end{equation}
\added{Compared to \eqref{eq:symmetric_timoshenko}, the formulation of the shear strain contribution to the stiffness matrix shown in \eqref{eq:nonsymmetric_timoshenko} requires one less matrix operation. One can note that, due to the different choices of the function spaces for the shear stress and its variation, the global stiffness matrix is no longer symmetric.}

\paragraph{Remark} 

\subsection{Bandwidth of the stiffness matrix}
A global $\bar{B}$ method that utilizes a global $L^2$ projection results in a dense stiffness matrix. The B\'ezier $\bar{B}$ \replaced{methods}{method}, on the other hand, \replaced{produce}{produces} sparse stiffness matrices. However, the coupling of the local dual basis functions does increase the bandwidth slightly. This is illustrated in Figure~\ref{fig:stiffness_matrix}, which shows the structure of the stiffness matrix for the Timoshenko beam problem using the second order basis functions of maximal smoothness for a displacement-based method (Figure~\ref{fig:stiffness_matrix}a), global $\bar{B}$ method (Figure~\ref{fig:stiffness_matrix}b), symmetric B\'ezier $\bar{B}$ method (Figure~\ref{fig:stiffness_matrix}c) and non-symmetric \Bezier $\bar{B}$ method (Figure~\ref{fig:stiffness_matrix}d). The blank cells indicate zero terms in the matrix while colored cells show the location of nonzero terms.
\begin{figure}[!htb]
    \centering
    \begin{subfigure}[b]{0.24\linewidth}        %% or \columnwidth
        \centering
        \includegraphics[width=\linewidth]{Global}
        \caption{Standard}
        %\label{}
    \end{subfigure}
    \begin{subfigure}[b]{0.24\linewidth}        %% or \columnwidth
        \centering
        \includegraphics[width=\linewidth]{B_extraction}
        \caption{Global $\bar{B}$}
        %\label{}
    \end{subfigure}
    \begin{subfigure}[b]{0.24\linewidth}        %% or \columnwidth
        \centering
        \includegraphics[width=\linewidth]{LB_extraction}
        \caption{Symmetric B\'ezier $\bar{B}$}
        %\label{}
    \end{subfigure}
    \begin{subfigure}[b]{0.24\linewidth}        %% or \columnwidth
        \centering
        \includegraphics[width=\linewidth]{NS-LB_extraction}
        \caption{Non-symmetric B\'ezier $\bar{B}$}
        %\label{}
    \end{subfigure}
    \caption{Illustrations of the structure of $2^{nd}$ order Timoshenko beam stiffness matrices for (a) a standard displacement method, (b) a global $\bar{B}$ method, (c) a symmetric B\'ezier $\bar{B}$ method and (d) a non-symmetric B\'ezier $\bar{B}$ method.}
    \label{fig:stiffness_matrix}
\end{figure}
∏
The increased bandwidth of the \added{symmetric} B\'ezier $\bar{B}$ method when compared to a displacement-based method can be explained by looking at the product of the integrals in \eqref{eqn:shear_stiffness}. For example, if we consider the basis functions $N_{1}$ and $N_{5}$ in Figure~\ref{fig:IGAelement} we see that $\operatorname{supp}(N_{1}) \cap{} \operatorname{supp}(N_{5}) = \emptyset$, which means that the inner product of these two functions will be zero and the corresponding coefficient in the stiffness matrix will be zero in the displacement-based method. For the B\'ezier $\bar{B}$ method, however, the form of \eqref{eqn:shear_stiffness} leads to a coupling between $N_{1}$ and $N_{5}$. This can be seen by considering $\Omega_2$. Over this element, the shear stiffness can be represented as
\begin{align}
	\bar{\mathbf{K}}^s_2 = \sum_{i=1}^3 \sum_{j=1}^3 \mathbf{P}_i^T\bar{\mathbf{M}}_2\mathbf{P}_j
\end{align}
and the term of this summation that results in the coupling between $N_{1}$ and $N_{5}$ is $\mathbf{P}_1^T\bar{\mathbf{M}}_2\mathbf{P}_3$, where $\mathbf{P}_1$ is the inner product of $N_{1}$ and $\hat{\bar{N}}_{2}$, $\mathbf{P}_3$ is the inner product of $N_{5}$ and $\hat{\bar{N}}_{3}$, and $\bar{\mathbf{M}}_2$ is the inner product of $\bar{N}_{2}$ and $\bar{N}_{3}$. We can see from Figure~\ref{fig:IGAelement} that $\operatorname{supp}(N_{1})\cap{}\operatorname{supp}(\hat{\bar{N}}_{2})=\Omega_1$, $\operatorname{supp}(N_{5})\cap{}\operatorname{supp}(\hat{\bar{N}}_{3})=\Omega_3$ and $\operatorname{supp}(\bar{N}_{2})\cap{}\operatorname{supp}(\bar{N}_{3})=\Omega_2$, so that $\mathbf{P}_1^T\bar{\mathbf{M}}_2\mathbf{P}_3$ is not zero. Thus we have increased the number of nonzero coefficients in the shear stiffness matrix. However, the same exercise can be used to show that there is no coupling between $N_{0}$ and $N_{6}$ for this set of basis functions so matrix is not dense. \added{With the abscense of the gramian matrix in the formulation of shear stiffness matrix, the bandwidth of the non-symmetric B\'ezier $\bar{B}$ method is even more reduced.} In fact, from the formulation of the element stiffness matrix, we can show that the bandwidth of the stiffness matrix of the symmetric B\'ezier $\bar{B}$ and non-symmetric B\'ezier $\bar{B}$ methods for the Timoshenko beam are $6p-3$ and $4p-1$, respectively. 

\paragraph{Remark} In \cite{bouclier_efficient_2013} a local $\bar{B}$ method for shells was proposed that was based on the local least squares method presented in \cite{govindjee_convergence_2012}. This approach has a similar structure to the \added{symmetric B\'ezier $\bar{B}$} method presented here. However, it was shown in~\cite{thomas_bezier_2015} that choosing (\ref{eqn:Bezier_weight}) as the weighting provides a significant increase in the accuracy of the approximation.

\begin{figure}[htb!]
      \centering
      \includegraphics[width=.5\linewidth]{basis_dual}
\caption{Quadratic maximally smooth B-spline basis functions (top), associated linear basis functions (middle), and dual basis functions (bottom) for the B\'ezier $\bar{B}$ formulation.}
\label{fig:IGAelement}
\end{figure}

\subsection{Numerical results}

In our study, a straight planar cantilever beam is clamped on the left end and a sinusoidal distributed load $f(x)=sin(\pi\dfrac{x}{l})$ is applied, as depicted in Figure \ref{fig:load_beam}. The analytical solution for vertical displacement $w$, rotation $\phi$, bending moment $M$, and transverse shear force $Q$ are given by
\begin{align}
\begin{split}
w(x)&=\frac{EI\left(6 \pi ^2 l^2 \sin \left(\frac{\pi  x}{l}\right)+6 \pi ^3 l x\right)+sGA\left(6 l^4 \sin \left(\frac{\pi  x}{l}\right)-6 \pi l^3 x+3 \pi ^3 l^2 x^2-\pi ^3 l x^3\right)}{6 \pi ^4 sEIGA}\\
\phi(x)&=\frac{2 l^3 \cos \left(\frac{\pi  x}{l}\right)-2 l^3+2 \pi ^2 l^2 x-\pi ^2 l x^2}{2 \pi ^3 EI}\\
M(x)&=\frac{l^2 \sin \left(\frac{\pi  x}{l}\right)-\pi  l^2+\pi  l x}{\pi ^2}\\
Q(x)&=\frac{-l\cos \left(\frac{\pi  x}{l}\right)-l}{\pi}.
\end{split}
\label{eq:timoshenko_analytical}
\end{align}
\begin{figure}[h]
	\centering
	\includegraphics[width=0.4\linewidth]{beam_load}
	\caption{Straight planar cantilevered Timoshenko beam clamped at the left and loaded by a distributed load $f(x)$.}
	\label{fig:load_beam}
\end{figure}

The beam has a rectangular cross-section and we use the following non-dimensional sectional and material parameters: length $l=10$, width $b=1$, thickness $t=0.01$, Young's modulus $E=10^9$, Poisson's ratio $\nu=0.3$, and a shear correction factor of $s=5/6$. A comparison of the normalized error in the $L^2$ norm for $w$, $\phi$, $M$ and $Q$ versus the number of degrees of freedom for polynomial degrees $p=1,2,3$ is shown in Figure \ref{fig:timoshenko_result}. Results computed using standard finite elements are labeled $Q_1$, $Q_2$, and $Q_3$. Results computed using a global $\bar{B}$ method are labeled $\mathcal{T}^{L^2}$, those computed with \replaced{the symmetric B\'ezier $\bar{B}$ method and the non-symmetric B\'ezier $\bar{B}$ method are labeled $S-\mathcal{T}^{P}$ and $NS-\mathcal{T}^{P}$, respectively. }{the B\'ezier $\bar{B}$ method are labeled $\mathcal{T}^{P}$}. As expected, the $Q_1$ results lock and the error remains virtually unchanged as the mesh is refined. Increasing the polynomial degree does reduce the locking effect, although the reduction is minor for the $Q_2$ results. \replaced{$\mathcal{T}^{L^2}$, $S-\mathcal{T}^{P}$ and $NS-\mathcal{T}^P$}{Both $\mathcal{T}^{L^2}$ and $\mathcal{T}^P$} are essentially locking free for all polynomial orders. The convergence rates for the $\bar{B}$ methods are at least $p+1$ for $w$, $p$ for $\phi$, $p-1$ for $M$, and $p-2$ for $Q$. These rates agree with those reported in \cite{kiendl_single-variable_2015} and are optimal. To reiterate, B\'{e}zier $\bar{B}$ methods produces the same convergence rates as the global $\bar{B}$ method \added{and the the error plots of $NS-\mathcal{T}^{P}$ for $\phi$, $M$ and $Q$ are identical to those of $\mathcal{T}^{L^2}$}.
\begin{figure}
    \centering
    \begin{subfigure}[b]{0.49\linewidth}        %% or \columnwidth
        \centering
        \includegraphics[width=\linewidth]{w-sin}
        \caption{}
        \vspace*{2mm}
    \end{subfigure}
    \begin{subfigure}[b]{0.49\linewidth}        %% or \columnwidth
        \centering
        \includegraphics[width=\linewidth]{phi-sin}
        \caption{}
        \vspace*{2mm}
    \end{subfigure}
        \begin{subfigure}[b]{0.49\linewidth}        %% or \columnwidth
        \centering
        \includegraphics[width=\linewidth]{M-sin}
        \caption{}
        \vspace*{2mm}
    \end{subfigure}
    \begin{subfigure}[b]{0.49\linewidth}        %% or \columnwidth
        \centering
        \includegraphics[width=\linewidth]{Q-sin}
        \caption{}
        \vspace*{2mm}
    \end{subfigure}
    \caption{Convergence studies for slenderness factor $l/t=10^{-3}$. Error in the $L^2$-norm for (a) displacement $w$, (b) rotation $\phi$, (c) bending moment $M$, and (d) shear force $Q$.}
    \label{fig:timoshenko_result}
\end{figure}

We have also studied the relationship between shear locking and decreasing slenderness ratios for $p=2$. The results are shown in Figure \ref{fig:slenderness}. For all three methods, the number of degrees of freedom are fixed, and the sectional and material parameters are the same as in the previous study. The slenderness ratio varies from $10$ to $5\times{10}^3$. $Q_2$ locks severely. The $\bar{B}$ methods, on the other hand, are locking free. 
%Note that the jump in the error for at the largest slenderness for the $\bar{B}$ methods is due to numerical round-off errors. If the computation is performed with high precision floating point numbers this jump is not present.

\begin{figure}
    \centering
    \begin{subfigure}[b]{0.49\linewidth}        %% or \columnwidth
        \centering
        \includegraphics[width=\linewidth]{slenderness_w}
        \caption{}
        \vspace*{2mm}
    \end{subfigure}
    \begin{subfigure}[b]{0.49\linewidth}        %% or \columnwidth
        \centering
        \includegraphics[width=\linewidth]{slenderness_phi}
        \caption{}
        \vspace*{2mm}
    \end{subfigure}
        \begin{subfigure}[b]{0.49\linewidth}        %% or \columnwidth
        \centering
        \includegraphics[width=\linewidth]{slenderness_M}
        \caption{}
        \vspace*{2mm}
    \end{subfigure}
    \begin{subfigure}[b]{0.49\linewidth}        %% or \columnwidth
        \centering
        \includegraphics[width=\linewidth]{slenderness_Q}
        \caption{}
        \vspace*{2mm}
    \end{subfigure}
    \caption{Convergence study for increasing slenderness, $p=2$, and $\#{dof}=32$. Error in the $L^2$-norm for (a) displacement $w$, (b) rotation $\phi$, (c) bending moment $M$, and (d) shear force $Q$.}
    \label{fig:slenderness}
\end{figure}


\section{Volumetric locking: Nearly incompressible linear elasticity}
\label{sec:nearlyincompressible}
To demonstrate the use of \Bezier $\bar{B}$ \replaced{methods}{method} to alleviate volumetric locking effects we study the nearly incompressible elasticity problem in two dimensions. We start with the small strain tensor $\boldsymbol{\varepsilon}$, which is defined as the symmetric part of the displacement gradient, i.e., 
\begin{equation}
\varepsilon_{ij}=\dfrac{u_{i,j}+u_{j,i}}{2}.
\end{equation}
The stress tensor is related to the strain tensor through the generalized Hooke's law
\begin{equation}
\sigma_{ij}=c_{ijkl}\varepsilon_{kl}
\end{equation}
where, for isotropic elasticity, the elastic coefficients and stress tensor can be expressed in terms of the Lam\'{e} parameters $\lambda$ and $\mu$ as
\begin{align}
c_{ijkl} &=\mu(\delta_{ik}\delta_{jl}+\delta_{il}\delta_{jk})+\lambda\delta_{ij}\delta_{kl}\\
\sigma_{ij} &= \lambda\varepsilon_{kk}\delta_{ij} + 2\mu\varepsilon_{ij}.
\end{align}
The Lam\'e parameters $\lambda$ and $\mu$ are defined in terms of Young's modulus, $E$, and Poisson's ratio, $\nu$, as
\begin{align}
\lambda &=\dfrac{\nu{E}}{(1+\nu)(1-2\nu)}\\
\mu &=\dfrac{\nu{E}}{2(1+\nu)}.
\end{align}
we can write the strong form of linear elasticity as
\begin{align}
	\sigma_{ij,j}+f_i &= 0 \text{ in $\Omega$} \\
    u_i &= g_i\text{ on $\Gamma_{g_i}$} \\
    \sigma_{ij}n_j &= h_i\text{ on $\Gamma_{h_i}$}
\end{align}
where Dirichlet boundary conditions are applied on $\Gamma_{g_i}$, Neumann boundary conditions are applied on $\Gamma_{h_i}$, and the closure of the domain $\Omega$ is $\bar{\Omega}=\Omega\cup\Gamma_{g_i}\cup\Gamma_{h_i}$. \added{To desmonstrate the source of volumetric locking, we introduce the pressure term}
\begin{equation}
    p=-(\lambda+2\mu/3){\epsilon}_{ii}.
\end{equation}
If $\nu\rightarrow\dfrac{1}{2}$ then $\lambda$ becomes very large and \replaced{an additional constraint is applied to the volumetric strain as}{standard finite element methods exhibit volumetric locking}.
\begin{equation}
    {\epsilon}_{ii}=0.
\end{equation}
\subsection{Symmetric \Bezier $\bar{B}$ projection}
\subsubsection{The weak form}
The $\bar{B}$ approach for nearly incompressible linear elasticity splits the strain tensor $\varepsilon$ into volumetric and deviatoric strains and then replaces the volumetric strain with a projected strain. We begin with
\begin{equation}
\boldsymbol{\varepsilon}(\mathbf{u})=\boldsymbol{\varepsilon}^{vol}(\mathbf{u})+\boldsymbol{\varepsilon}^{dev}(\mathbf{u})
\end{equation}
where $\boldsymbol{\varepsilon}^{vol}=\dfrac{1}{3}\mathrm{tr}(\boldsymbol{\varepsilon})\mi$ is the volumetric strain and $\boldsymbol{\varepsilon}^{dev}=\boldsymbol{\varepsilon}-\dfrac{1}{3}\mathrm{tr}(\boldsymbol{\varepsilon})\mi$ is the deviatoric strain. The volumetric strain is then replaced by a projected volumetric strain $\bar{\boldsymbol{\varepsilon}}^{vol}$ and the new total strain becomes
\begin{equation}
\bar{\boldsymbol{\varepsilon}}=\bar{\boldsymbol{\varepsilon}}^{vol}+\boldsymbol{\varepsilon}^{dev}.
\end{equation}
The weak form can then be written as: find $\mathbf{u}\in{\mathcal{S}(\Omega)}$ such that for all $\mathbf{w}\in{\mathcal{V}(\Omega)}$
\begin{equation}
    \bar{a}\langle{\mathbf{w},\mathbf{u}}\rangle=\bar{l}\langle{\mathbf{w}}\rangle
\end{equation}
where
\begin{align}
    \bar{a}\langle{\mathbf{w},\mathbf{u}}\rangle&=\int_{\Omega}\bar{\epsilon}_{ij}(\mathbf{w})c_{ijkl}\bar{\epsilon}_{kl}(\mathbf{u})d\Omega, \\
    \bar{l}\langle{\mathbf{w}}\rangle&=\int_{\Omega}\mathbf{w}\cdot\mathbf{f}d\Omega+\int_{\Gamma_{h}}\mathbf{w}\cdot\mathbf{h}d\Gamma.
\end{align}

\subsubsection{Discretization}
Following the same approach as was described for Timoshenko beams in Section~\ref{sec:timoshenkobeam} we define element level strain-displacement matrices in terms of the Bernstein basis 
\begin{align}
\mathbf{B}^{dev}_e&=
\begin{bmatrix}
\dfrac{2}{3}\dfrac{\partial{B}^e_{0,p}}{\partial{x}} & -\dfrac{1}{3}\dfrac{\partial{B}^e_{0,p}}{\partial{y}} & -\dfrac{1}{3}\dfrac{\partial{B}^e_{0,p}}{\partial{z}} & \cdots & \dfrac{2}{3}\dfrac{\partial{B}^e_{p,p}}{\partial{x}} & -\dfrac{1}{3}\dfrac{\partial{B}^e_{p,p}}{\partial{y}} & -\dfrac{1}{3}\dfrac{\partial{B}^e_{p,p}}{\partial{z}} \\
-\dfrac{1}{3}\dfrac{\partial{B}^e_{0,p}}{\partial{x}}  & \dfrac{2}{3}\dfrac{\partial{B}^e_{0,p}}{\partial{y}} & -\dfrac{1}{3}\dfrac{\partial{B}^e_{0,p}}{\partial{z}} & \cdots & -\dfrac{1}{3}\dfrac{\partial{B}^e_{p,p}}{\partial{x}}  & \dfrac{2}{3}\dfrac{\partial{B}^e_{p,p}}{\partial{y}} & -\dfrac{1}{3}\dfrac{\partial{B}^e_{p,p}}{\partial{z}}\\
-\dfrac{1}{3}\dfrac{\partial{B}^e_{0,p}}{\partial{x}} & -\dfrac{1}{3}\dfrac{\partial{B}^e_{0,p}}{\partial{y}} & \dfrac{2}{3}\dfrac{\partial{B}^e_{0,p}}{\partial{z}} & \cdots & -\dfrac{1}{3}\dfrac{\partial{B}^e_{p,p}}{\partial{x}} & -\dfrac{1}{3}\dfrac{\partial{B}^e_{p,p}}{\partial{y}} & \dfrac{2}{3}\dfrac{\partial{B}^e_{p,p}}{\partial{z}}\\
0 & \dfrac{\partial{B}^e_{0,p}}{\partial{z}} & \dfrac{\partial{B}^e_{0,p}}{\partial{y}} & \cdots & 0 & \dfrac{\partial{B}^e_{p,p}}{\partial{z}} & \dfrac{\partial{B}^e_{p,p}}{\partial{y}}\\
\dfrac{\partial{B}^e_{0,p}}{\partial{z}} & 0 & \dfrac{\partial{B}^e_{0,p}}{\partial{x}} & \cdots & \dfrac{\partial{B}^e_{p,p}}{\partial{z}} & 0 & \dfrac{\partial{B}^e_{p,p}}{\partial{x}}\\
\dfrac{\partial{B}^e_{0,p}}{\partial{y}} & \dfrac{\partial{B}^e_{0,p}}{\partial{x}} & 0 & \cdots & \dfrac{\partial{B}^e_{p,p}}{\partial{y}} & \dfrac{\partial{B}^e_{p,p}}{\partial{x}} & 0\\
\end{bmatrix},
\end{align}
\begin{align}
\mathbf{B}_e^{vol}&=
\begin{bmatrix}
\dfrac{\partial{B}^e_{0,p}}{\partial{x}} & \dfrac{\partial{B}^e_{0,p}}{\partial{y}} & \dfrac{\partial{B}^e_{0,p}}{\partial{z}} & \cdots & \dfrac{\partial{B}^e_{p,p}}{\partial{x}} & \dfrac{\partial{B}^e_{p,p}}{\partial{y}} & \dfrac{\partial{B}^e_{p,p}}{\partial{z}}
\end{bmatrix}
\end{align}
The deviatoric part of the element stiffness matrix can then be computed from the corresponding strain-displacement matrices as
\begin{align}
	\mathbf{K}_e^{dev}
    	=\mathbf{C}^e\langle {\mathbf{B}_{e}^{dev}}^T\mathbf{D}{\mathbf{B}}_{e}^{dev}\rangle (\mathbf{C}^e)^T.
\end{align}
where $\mathbf{C}^e$ is the element extraction operator for the degree $p$ spline space. The volumetric part of the stiffness matrix is computed using Bezier projection. The intermediate element matrices are
\begin{align}
\bar{\mathbf{M}}^{vol}_e&=\bar{\mathbf{C}}^e\langle{\bar{\mathbf{B}}_e^T,\dfrac{1}{3}(3\lambda+2\mu)\bar{\mathbf{B}}_e}\rangle(\bar{\mathbf{C}}^e)^T\\
\hat{\mathbf{P}}^{vol}_e&=\langle{(\hat{\bar{\mathbf{N}}}^e)^T, \mathbf{B}_e^{vol}}\rangle(\mathbf{C}^e)^T
\end{align}
where $\bar{\mathbf{C}}^e$ is the element extraction operator for the degree $p-1$ spline space, $\hat{\bar{\mathbf{N}}}^e$ are the dual basis functions restriced to the element, and \begin{equation}
\bar{\mathbf{B}}_e=
\begin{bmatrix}
{B}^e_{0,p-1} & \cdots & {B}^e_{p-1,p-1}
\end{bmatrix}.
\end{equation}The global stiffness matrix can then be assembled as
\begin{align}
	\mathbf{K} = \mathbf{K}^{dev} + \bar{\mathbf{K}}^{vol}_{S}
\end{align}
where
\begin{align}
	\mathbf{K}^{dev} &=\A_e\mathbf{K}_e^{dev}, \\
	\bar{\mathbf{K}}_S^{vol}&=\hat{\mathbf{P}}^T\bar{\mathbf{M}}\hat{\mathbf{P}} \\
	\hat{\mathbf{P}} &=\A_e\hat{{\mathbf{P}}}^{vol}_e \\
    \bar{\mathbf{M}} &=\A_e\bar{\mathbf{M}}^{vol}_e.
\end{align}
\subsection{Non-symmetric \Bezier $\bar{B}$ projection}
\subsubsection{The weak form}
\added{In the context of nearly incompressible elasticity, the pressure $p$ can be viewed as an independent unknown, leading to a mixed formulation, as: find $\mathbf{u}\in{\mathcal{S}(\Omega)}$ and $p\in{L}^2(\Omega)$ such that for all $\mathbf{w}\in{\mathcal{V}(\Omega)}$ and $\delta{p}\in{L}^2(\Omega)$}
\begin{align}
    \hat{a}\langle{\mathbf{w},\mathbf{u}}\rangle-b\langle{\mathbf{w},p}\rangle&=l\langle{\mathbf{w}}\rangle, \\
    - b\langle{\mathbf{u},\delta{p}}\rangle-\dfrac{1}{(\lambda+2\mu/3)}\int_{\Omega}\delta{p}pd\Omega&=0,
\end{align}
where
\begin{align}
    \hat{a}\langle{\mathbf{w},\mathbf{u}}\rangle&=\int_{\Omega}\epsilon_{ij}(\mathbf{w})\hat{c}_{ijkl}\epsilon_{kl}(\mathbf{u})d\Omega,\\
    \hat{c}_{ijkl}&=\mu\left(\delta_{ik}\delta_{jl}+\delta_{il}\delta_{jk}-2/3\delta_{ij}\delta_{kl}\right),\\
    b\langle{\mathbf{w},p}\rangle&=\int_{\Omega}\epsilon_{ii}(\mathbf{w})pd\Omega.
\end{align}

\subsubsection{\added{Discretization}}

Following the same pattern of non-symmetric \Bezier $\bar{B}$ in Timoshenko beam problem, we keep the discretization of $\mathbf{u}$ and $\mathbf{w}$ untouched, while use lower order spline basis functions and corresponding dual basis functions for the discretization of pressure $p$ and its variation $\delta{p}$, respectively. The discretized stiffness matrix in mixed form can be written as
\begin{equation}
    \mathbf{K}_{mix}=
    \begin{bmatrix}
        \mathbf{K}^{dev} & -\mathbf{P}^{T}\\
        -\hat{\mathbf{P}} & -\dfrac{1}{(\lambda+2\mu/3)}\mathbf{I}
    \end{bmatrix},
\end{equation}
\added{where}
\begin{equation}
    \mathbf{P}=\A_e\mathbf{P}_e^{vol},
\end{equation}
with
\begin{equation}
    \mathbf{P}_e^{vol}=\bar{\mathbf{C}}^e\langle{\bar{\mathbf{B}}_e^T,\mathbf{B}^{vol}_e}\rangle(\mathbf{C}^e)^T.
\end{equation}
By eliminating the pressure control points, we obtain the pure displacement formulation as
\begin{equation}
	\mathbf{K} = \mathbf{K}^{dev} + \bar{\mathbf{K}}^{vol}_{NS},
\end{equation}
where
\begin{equation}
    \bar{\mathbf{K}}^{vol}_{NS}=(\lambda+2\mu/3){\mathbf{P}}^T\hat{\mathbf{P}}.
\end{equation}

\subsection{Numerical results}
We investigate the performance of the B\'{e}zier $\bar{B}$ method for two nearly incompressible linear elasticity problems under plane strain conditions. \added{Previous to the examples, we numerically evaluate the inf–sup constant for global $\bar{B}$ and non-symmetric \Bezier $\bar{B}$ methods. For numerical examples,} we first study the Cook's membrane problem, which is discretized with B-spline basis functions, and in the second problem we model the infinite plate with a circular hole problem using NURBS. Results computed using standard finite elements are labeled $Q_1$, $Q_2$, $Q_3$, and $Q_4$. Results computed using a global $\bar{B}$ method are labeled $\mathcal{T}^{L^2}$ and those computed with the \added{symmetric} B\'ezier $\bar{B}$ method and the \added{non-symmetric B\'ezier $\bar{B}$ method} are labeled $S-\mathcal{T}^{P}$ and $NS-\mathcal{T}^{P}$, respectively.

\subsubsection{\added{Numerical evaluation of the inf-sup condition}}

The inf-sup condition is also refered to as the Ladyzhenskaya-Babuska-Brezzi condition (or simply LBB). It is a crucial condition to ensure the solvability, stability and optimality of a mixed problem. For the nearly incompressible elasticity problem, the inf-sup condition is, for $\delta{p}\neq{0}$ and $\mathbf{u}\neq{0}$
\begin{equation}
    \inf_{\delta{p}\in{L^2(\Omega)}}\sup_{\mathbf{u}\in\mathcal{S}(\Omega)}\dfrac{\vert{b\langle{\mathbf{u},\delta{p}}\rangle}\vert}{\|{\delta{p}}\|_{L^2(\Omega)}\|{\mathbf{u}}\|_{H^1(\Omega)}}\geq\beta>0.
\end{equation}
In a discretized problem, the inf-sup condition requires the variable $\beta$ to be a constant that is independent of the mesh size.\par

Here, we consider the inf-sup condition of a uniformly refined quarter annulus. The geometry and boundary conditions are demonstrated in Figure \ref{fig:quarter_annulus}. The geometry of the quarter annulus can be exactly represented using a biquadratic NURBS basis. The knot vector for the coarsest discretization is given by
\begin{align}
\begin{split}
\Xi_\xi\times\Xi_\eta=\lbrace{0,0,0,1,1,1}\rbrace\times\lbrace{0,0,0,1,1,1}\rbrace
\end{split}
\end{align}
and the corresponding weights and control points associated with each basis function are given in Table \ref{table:weights} and \ref{table:control_points}. For higher-order elements and finer discretizations the weights and corresponding control points are identified by an order elevation and knot insertion algorithm, respectively. The B\'ezier mesh representation for the discretizations are shown in Figure \ref{fig:mesh_hole}.

\begin{figure}[htb!]
	\centering
	\includegraphics[width=0.5\linewidth]{annular}
	\caption{Geometry and boundary conditions for the inf-sup test.}
	\label{fig:quarter_annulus}
\end{figure}

\begin{table}
\begin{center}
\caption{Weights for the plate with a circular hole}\label{table:weights}
\begin{tabular}{l@{\hskip 1cm}l@{\hskip 1cm}l@{\hskip 1cm}l}
\hline
i    & $w_{i,1} $ & $w_{i,2}$ & $w_{i,3}$\\
\hline
1    & 1    & ${1}/{\sqrt{2}}$ & 1 \\
2    & 1    & ${1}/{\sqrt{2}}$ & 1 \\
3    & 1    & ${1}/{\sqrt{2}}$ & 1 \\
\hline
\end{tabular}
\end{center}
\end{table}

\begin{table}
\begin{center}
\caption{Control points for the plate with a circular hole}\label{table:control_points}
\begin{tabular}{l@{\hskip 1cm}l@{\hskip 1cm}l@{\hskip 1cm}l}
\hline
i    & $B_{i,1} $ & $B_{i,2}$ & $B_{i,3}$\\
\hline
1    & (0,1)    & (1,1)     & (1,0)   \\
2    & (0,2.5)  & (2.5,2.5) & (2.5,0) \\
3    & (0,4)    & (4,4)     & (4,0)   \\
\hline
\end{tabular}
\end{center}
\end{table}
\begin{figure}
    \centering
    \begin{subfigure}[b]{0.18\linewidth}        %% or \columnwidth
        \centering
        \includegraphics[width=\linewidth]{mesh_hole_0}
    \end{subfigure}
    \begin{subfigure}[b]{0.18\linewidth}        %% or \columnwidth
        \centering
        \includegraphics[width=\linewidth]{mesh_hole_1}
    \end{subfigure}
    \begin{subfigure}[b]{0.18\linewidth}        %% or \columnwidth
        \centering
        \includegraphics[width=\linewidth]{mesh_hole_2}
    \end{subfigure}
    \begin{subfigure}[b]{0.18\linewidth}        %% or \columnwidth
        \centering
        \includegraphics[width=\linewidth]{mesh_hole_3}
    \end{subfigure}
    \begin{subfigure}[b]{0.18\linewidth}        %% or \columnwidth
        \centering
        \includegraphics[width=\linewidth]{mesh_hole_4}
    \end{subfigure}
    \caption{Sequence of meshes for inf-sup test.}
    \label{fig:mesh_hole}
\end{figure}


Only the global $\bar{B}$ method and the non-symmetric \Bezier $\bar{B}$ method are considered here, as the symmetric \Bezier $\bar{B}$ method lacks connection to the mixed formulation. As a counter example, the well-known pair $Q_p/Q_p$ of global $\bar{B}$ that violates the inf-sup conditon is also tested here. Our tests follow the procedure proposed by Chapelle and Bathe in \cite{chapelle_inf-sup_1993}.\par
\begin{figure}[htb!]
    \centering
    \begin{subfigure}[b]{0.31\linewidth}        %% or \columnwidth
        \centering
        \includegraphics[width=\linewidth]{beta_p_2}
    \end{subfigure}
    \begin{subfigure}[b]{0.31\linewidth}        %% or \columnwidth
        \centering
        \includegraphics[width=\linewidth]{beta_p_3}
    \end{subfigure}
    \begin{subfigure}[b]{0.31\linewidth}        %% or \columnwidth
        \centering
        \includegraphics[width=\linewidth]{beta_p_4}
    \end{subfigure}

    \caption{Inf-sup test results for nearly incompressible elasticity. The stability parameter $\beta$ with respect to mesh size.}
    \label{fig:inf_sup}
\end{figure}

Figure \ref{fig:inf_sup} shows the numerical results. As can be seen, the global $\bar{B}$ method and the non-symmetric \Bezier $\bar{B}$ method are not strictly satisfying the LBB condition, which is consistent with the statement that the global $\bar{B}$ method does not reduce the constraints sufficiently to satisfy LBB condition in \cite{elguedj:hal-00457010}. But compared to the $Q_p/Q_p$ pair, both methods reduce constraints to a favorable level. Comparing the global $\bar{B}$ method with non-symmetric \Bezier $\bar{B}$ method, their stability parameter $\beta$ decrease in the same rate and the stability parameter for the non-symmetric \Bezier $\bar{B}$ method is slightly lower than that for the global $\bar{B}$ method, which indicate a similar optimality in convergence for both methods and a slightly higher error for the non-symmetric \Bezier $\bar{B}$ method.

\subsubsection{Cook's membrane problem}

This benchmark problem is a standard test for combined bending and shearing response. The geometry, boundary conditions, and material properties are shown in Figure \ref{fig:Cook's}. The left boundary of the tapered panel is clamped, the top and bottom edges are free with zero traction boundary conditions, and the right boundary is subjected to a uniformly distributed traction load in the $y$-direction as shown. The meshes used are shown in Figure \ref{fig:mesh_cook}.
\begin{figure}[htb!]
	\centering
	\includegraphics[width=0.5\linewidth]{Cook_s}
	\caption{Geometry, boundary conditions, and material properties for the Cook's membrane problem.}
	\label{fig:Cook's}
\end{figure}

A comparison of the displacement of the top right corner with respect to the number of elements per side is shown in Figure \ref{fig:Cook's_result}. $Q_1$ locks and mesh refinement has little impact. Locking is somewhat reduced for the higher-order elements $Q_p$, $p > 1$. The $\bar{B}$ methods perform very well for all degrees.
\begin{figure}[htb!]
    \centering
    \begin{subfigure}[b]{0.18\linewidth}        %% or \columnwidth
        \centering
        \includegraphics[width=\linewidth]{mesh_cook_0}
    \end{subfigure}
    \begin{subfigure}[b]{0.18\linewidth}        %% or \columnwidth
        \centering
        \includegraphics[width=\linewidth]{mesh_cook_1}
    \end{subfigure}
    \begin{subfigure}[b]{0.18\linewidth}        %% or \columnwidth
        \centering
        \includegraphics[width=\linewidth]{mesh_cook_2}
    \end{subfigure}
    \begin{subfigure}[b]{0.18\linewidth}        %% or \columnwidth
        \centering
        \includegraphics[width=\linewidth]{mesh_cook_3}
    \end{subfigure}
    \begin{subfigure}[b]{0.18\linewidth}        %% or \columnwidth
        \centering
        \includegraphics[width=\linewidth]{mesh_cook_4}
    \end{subfigure}
    \caption{Sequence of meshes for Cook's membrane problem.}
    \label{fig:mesh_cook}
\end{figure}

\begin{figure}[htb!]
    \centering
	\begin{subfigure}[b]{\textwidth}
        \includegraphics[width=.8\linewidth]{Cook's_membrane_Q1}
    \end{subfigure}
    %
    \centering
	\begin{subfigure}[b]{\textwidth}
        \includegraphics[width=.8\linewidth]{Cook's_membrane_Q2}
    \end{subfigure}
\end{figure}

\begin{figure}[htb!]\ContinuedFloat
    \centering
    \begin{subfigure}[b]{\textwidth}
        \includegraphics[width=.8\linewidth]{Cook's_membrane_Q3}
    \end{subfigure}
    %
    \centering
     \begin{subfigure}[b]{\textwidth}
        \includegraphics[width=.8\linewidth]{Cook's_membrane_Q4}
    \end{subfigure}
	\caption{Cook's membrane: comparison of the vertical displacement at the top right corner for the different methods and degrees.}
	\label{fig:Cook's_result}
\end{figure}

\clearpage

\subsubsection{Infinite plate with a circular hole}

The setup for the infinite plate with a circular hole problem is shown in Figure \ref{fig:platewithhole_geometry} and the discretizations are shown in Figure \ref{fig:mesh_hole}. The traction along the outer edge is evaluated from the exact solution which is given by
\begin{align}
\begin{split}
\sigma_{rr}(r,\theta)&=\dfrac{T_x}{2}(1-\dfrac{R_1^2}{r^2})+\dfrac{T_x}{2}(1-4\dfrac{R_1^2}{r^2}+3\dfrac{R_1^4}{r^4})cos(2\theta)\\
\sigma_{\theta\theta}(r,\theta)&=\dfrac{T_x}{2}(1+\dfrac{R_1^2}{r^2})-\dfrac{T_x}{2}(1+3\dfrac{R_1^4}{r^4})cos(2\theta)\\
\sigma_{r\theta}(r,\theta)&=-\dfrac{T_x}{2}(1+2\dfrac{R_1^2}{r^2}-3\dfrac{R_1^4}{r^4})sin(2\theta).
\end{split}
\end{align}
\begin{figure}[htb!]
	\centering
	\begin{subfigure}[t]{0.5\textwidth}
    \centering
    \includegraphics[width=1\linewidth]{Platewithhole}
    \caption{Infinite plate with a hole subjected to uniaxial tension at $x=\pm\infty$.}
    \label{fig:platewithhole_geometry_a}
    \end{subfigure}%
    ~ 
    \begin{subfigure}[t]{0.5\textwidth}
    \centering
    \includegraphics[width=1\linewidth]{Platewithhole_b}
    \caption{A representation of the computational model.}
    \label{fig:platewithhole_geometry_b}
    \end{subfigure}
    \caption{Geometry, boundary conditions, and material properties for the infinite plate with a hole.}
	\label{fig:platewithhole_geometry}
\end{figure}



Convergence plots for the relative error of the displacement and energy in the $L^2$ norm are shown in Figure \ref{fig:platewithhole_convergence}. As can be seen, the standard $Q_p$ approximations suffer from severe volumetric locking for all orders while, on the other hand, the $\bar{B}$ methods remedy locking for all cases. \replaced{For the symmetric \Bezier $\bar{B}$ method, the optimality in all three measures has been obtained for biquadratic elements; the optimal energy convergence rate has been achieved for bicubic elements; however, the degradation of convergence rates in all three measures has been observed for biquartic elements. The can be attributed to the fact that the derivation of the symmetric \Bezier $\bar{B}$ method is purely based on the engineering analogy between the $L^2$ projection and \Bezier projection. On the other hand, the non-symmetric \Bezier $\bar{B}$ method, derived from the mixed formulation, is optimal for displacement, stress and energy for all elements with slightly higher errors than that of global $\bar{B}$ method, which agrees with the inf-sup test.}{For biquadratic elements, the \Bezier $\bar{B}$ method obtains optimal convergence rates for both the displacement and energy error, and the difference in the energy error between the global $\bar{B}$ and the B\'ezier $\bar{B}$ methods is indistinguishable. For bicubic and biquartic elements, the convergence rates of the B\'ezier $\bar{B}$ method for the energy error are optimal. The convergence rates of B\'ezier $\bar{B}$ method for the displacement error, however, are not optimal. This degradation in the rates can likely be attributed to the conditioning of the Bernstein basis and the element extraction operators.}\par

\added{The error contour plots of $\sigma_{xx}$ for biquartic elements of the fifth mesh are shown in Figure \ref{fig:platewithhole_contour}, $Q_4$ elements are severely locked while all $\bar{B}$ methods converge to the reference solution. }
\begin{figure}[h]
    \center
\begin{tabular}{ccc}
\includegraphics[width=.31\linewidth]{deformation_p_2} & \includegraphics[width=.31\linewidth]{deformation_p_3} & \includegraphics[width=.31\linewidth]{deformation_p_4}\\
\includegraphics[width=.31\linewidth]{stress_p_2} & \includegraphics[width=.31\linewidth]{stress_p_3} & \includegraphics[width=.31\linewidth]{stress_p_4}\\
\includegraphics[width=.31\linewidth]{energy_p_2} & \includegraphics[width=.31\linewidth]{energy_p_3} & \includegraphics[width=.31\linewidth]{energy_p_4}
\end{tabular}
    \caption{Convergence study of the plate with a circular hole. The relative $L^2$ error of displacement, stress and the relative error in energy norm with respect to mesh refinement.}
	\label{fig:platewithhole_convergence}
\end{figure}

\begin{figure}[htb!]
    \center
    \begin{tabular}{cc}
    \includegraphics[width=.4\linewidth]{no} & \includegraphics[width=.4\linewidth]{g}\\
    $Q_4$ & $\mathcal{T}^{L^2} Q_4/Q_3$\\
    \includegraphics[width=.4\linewidth]{s} & \includegraphics[width=.4\linewidth]{ns}\\
    $S-\mathcal{T}^{p} Q_4/Q_3$ & $NS-\mathcal{T}^{p} Q_4/Q_3$
    \end{tabular}
        \caption{Contour plots of $\vert{\sigma_{xx}-\sigma_{xx}^h}\vert$ for the plate with a circular hole ($p=4$, the $5^{th}$ mesh is used).}
        \label{fig:platewithhole_contour}
    \end{figure}
\section{Conclusions}
\label{sec:conclusion}

We have presented the B\'{e}zier $\bar{B}$ method as an approach to overcome locking phenomena in structural mechanics applications of isogeometric analysis. The approach utilizes B\'ezier extraction and projection which makes it simple to implement in an existing finite element framework and makes it applicable to any spline representation which can be written in \Bezier form. In contrast to global $\bar{B}$ methods, which produce dense stiffness matrices, the B\'ezier $\bar{B}$ approach results in a sparse stiffness matrix while still benefiting from higher-order convergence rates.

We have demonstrated the performance of the approach in the context of shear deformable beams (to alleviate transverse shear locking) and nearly incompressible elasticity problems (to alleviate volumetric locking). The proposed method reduces locking errors and achieves (nearly) optimal convergence rates. The cases where optimal rates were not achieved warrant further study.
\bibliographystyle{elsarticle-num}
\bibliography{bibliography}
\end{document}
